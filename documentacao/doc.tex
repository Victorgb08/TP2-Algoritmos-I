\documentclass[a4paper,12pt]{article}

% Pacotes necessários
\usepackage[utf8]{inputenc} % Codificação UTF-8
\usepackage[T1]{fontenc}    % Suporte para caracteres acentuados
\usepackage[portuguese]{babel} % Idioma português
\usepackage{geometry}       % Configuração de margens
\usepackage{hyperref}       % Links clicáveis no documento
\usepackage{amsmath}        % Pacote para fórmulas matemáticas

% Configuração de margens
\geometry{a4paper, margin=2.5cm}

\begin{document}

% Título e informações do autor
\begin{center}
    \Large\textbf{Documentação do Trabalho Prático 2} \\
    \large Algoritmos I - 2025/1 \\[1em]
    Victor Guedes Batista \\ Matrícula: 2020070817 \\[1em]
    \today
\end{center}

% Capítulo 1: Introdução
\section*{Introdução}

Este documento apresenta a solução para o \textbf{Trabalho Prático 2} da disciplina de Algoritmos I, cujo objetivo é determinar o número mínimo de soldados necessários para proteger a capital de um reino fictício contra invasores.

O problema é modelado como um grafo, onde cada célula do mapa representa um custo de defesa. A solução utiliza algoritmos de fluxo máximo para calcular o corte mínimo que isola a capital do restante do mapa.

Os capítulos seguintes detalham a modelagem, a solução implementada, a análise de complexidade e as considerações finais.

% Capítulo 2: Modelagem
\section*{Modelagem}

O problema foi modelado como um grafo direcionado, onde cada célula do mapa é representada por dois nós: \textit{vin} (entrada) e \textit{vout} (saída). Essa separação permite definir capacidades específicas para cada célula e suas conexões.

\subsection*{Representação do Grafo}

\begin{itemize}
    \item Cada célula é conectada de \textit{vin} para \textit{vout} com capacidade igual ao custo de defesa.
    \item Células vizinhas são conectadas de \textit{vout} para \textit{vin} com capacidade infinita.
    \item Um nó \textit{source} conecta as bordas do mapa (exceto montanhas) com capacidade infinita.
    \item O nó \textit{vout} da capital é conectado ao \textit{sink} com capacidade infinita.
\end{itemize}

\subsection*{Estrutura de Dados}

O grafo foi representado por uma matriz de adjacência com $2 \times (n \times m) + 2$ nós, onde $n$ e $m$ são as dimensões do mapa. Os índices dos nós foram calculados como:
\begin{itemize}
    \item \textit{vin}: $2 \times (i \times m + j)$
    \item \textit{vout}: $2 \times (i \times m + j) + 1$
\end{itemize}

\subsection*{Algoritmos Utilizados}

Para resolver o problema, foi utilizado o algoritmo de \textbf{Edmonds-Karp},
uma implementação do algoritmo de fluxo máximo de Ford-Fulkerson baseada em
busca em largura (BFS). A seguir, descrevemos os principais passos do
algoritmo:

\begin{enumerate}
    \item Inicializar o grafo residual com as capacidades fornecidas.
    \item Enquanto houver um caminho aumentador da \textit{source} para o \textit{sink}
          no grafo residual:
          \begin{enumerate}
              \item Encontrar o caminho aumentador utilizando BFS.
              \item Determinar a capacidade de gargalo do caminho.
              \item Atualizar as capacidades residuais ao longo do caminho.
          \end{enumerate}
    \item O fluxo máximo é a soma das capacidades de todos os caminhos aumentadores
          encontrados.
\end{enumerate}

\subsection*{Tradução do Problema}

A tradução do problema para o grafo pode ser resumida nos seguintes passos:
\begin{enumerate}
    \item Cada célula do mapa é transformada em dois nós (\textit{vin} e \textit{vout}).
    \item As conexões entre células vizinhas são representadas por arestas com capacidade
          infinita.
    \item As células nas bordas do mapa são conectadas à \textit{source}.
    \item A célula da capital é conectada ao \textit{sink}.
\end{enumerate}

Essa modelagem permite que o problema seja resolvido como um problema de fluxo
máximo, onde o corte mínimo no grafo corresponde ao número mínimo de soldados
necessários para proteger a capital.

% Capítulo 3: Solução
\section*{Solução}

A solução utiliza o algoritmo de \textbf{Edmonds-Karp} para calcular o fluxo máximo no grafo modelado. O fluxo máximo corresponde ao custo do corte mínimo, que é o número mínimo de soldados necessários para proteger a capital.

\subsection*{Passos da Solução}

\begin{enumerate}
    \item \textbf{Construção do Grafo}: Cada célula é representada por dois nós (\textit{vin} e \textit{vout}), conectados com base nas regras do problema.
    \item \textbf{Cálculo do Fluxo Máximo}: O algoritmo de Edmonds-Karp encontra caminhos aumentadores no grafo residual e atualiza as capacidades.
    \item \textbf{Interpretação do Resultado}: O valor do fluxo máximo é o custo do corte mínimo.
\end{enumerate}

A implementação foi realizada em Python, utilizando uma matriz de adjacência para representar o grafo e BFS para encontrar caminhos aumentadores.

\subsection*{Pseudocódigo do Algoritmo}

A seguir, apresentamos o pseudocódigo do algoritmo de Edmonds-Karp utilizado na
solução:

\begin{verbatim}
EdmondsKarp(G, source, sink):
    Inicializar fluxo máximo como 0
    Enquanto houver um caminho aumentador no grafo residual:
        Encontrar o caminho aumentador usando BFS
        Determinar a capacidade de gargalo do caminho
        Atualizar as capacidades residuais ao longo do caminho
        Adicionar a capacidade de gargalo ao fluxo máximo
    Retornar o fluxo máximo
\end{verbatim}

\subsection*{Detalhes da Implementação}

A implementação foi realizada em Python e segue os seguintes passos:

\begin{itemize}
    \item A função \texttt{parse\_input\_corrected} lê os dados de entrada e converte as
          coordenadas da capital para índices baseados em zero.
    \item A matriz de adjacência \texttt{capacity\_adj\_matrix} é construída para
          representar o grafo, com base na modelagem descrita no capítulo anterior.
    \item A função \texttt{edmonds\_karp\_max\_flow} calcula o fluxo máximo entre o
          \textit{source} e o \textit{sink}, utilizando BFS para encontrar caminhos
          aumentadores.
    \item A função \texttt{solve} integra todas as etapas, retornando o número mínimo de
          soldados necessários para proteger a capital.
\end{itemize}

Essa abordagem garante que o problema seja resolvido de forma eficiente e
correta, utilizando conceitos sólidos de teoria dos grafos.

% Capítulo 4: Análise de Complexidade
\section*{Análise de Complexidade}

Neste capítulo, analisamos a complexidade assintótica de tempo e memória da
solução implementada para o problema proposto. A análise considera as três
etapas principais da solução: construção do grafo, execução do algoritmo de
Edmonds-Karp e interpretação do resultado.

\subsection*{Construção do Grafo}

A construção do grafo envolve a criação de uma matriz de adjacência para
representar as capacidades entre os nós. Cada célula do mapa é representada por
dois nós (\textit{vin} e \textit{vout}), e as conexões entre células vizinhas
são adicionadas com base nas regras do problema.

\begin{itemize}
    \item \textbf{Tempo}: Para cada célula do mapa, verificamos seus vizinhos (até 4 vizinhos por célula). Assim, o tempo necessário para construir o grafo é proporcional ao número de células no mapa, ou seja, $O(n \times m)$, onde $n$ é o número de linhas e $m$ é o número de colunas.
    \item \textbf{Memória}: A matriz de adjacência possui $2 \times (n \times m) + 2$ nós, resultando em um espaço de $O((n \times m)^2)$ para armazenar as capacidades.
\end{itemize}

\subsection*{Execução do Algoritmo de Edmonds-Karp}

O algoritmo de Edmonds-Karp é uma implementação do algoritmo de Ford-Fulkerson
que utiliza busca em largura (BFS) para encontrar caminhos aumentadores no
grafo residual.

\begin{itemize}
    \item \textbf{Tempo}: O algoritmo realiza até $O(E)$ iterações de BFS, onde $E$ é o número de arestas no grafo. Cada BFS percorre no máximo $O(V)$ nós, onde $V$ é o número de nós no grafo. Assim, a complexidade total do algoritmo é $O(V \times E^2)$. No caso do grafo construído, temos $V = 2 \times (n \times m) + 2$ e $E = O(n \times m)$, resultando em uma complexidade de $O((n \times m)^3)$.
    \item \textbf{Memória}: O grafo residual é armazenado como uma matriz de adjacência, ocupando $O((n \times m)^2)$ de espaço.
\end{itemize}

\subsection*{Interpretação do Resultado}

A interpretação do resultado consiste em retornar o valor do fluxo máximo
calculado pelo algoritmo de Edmonds-Karp. Essa etapa é constante em termos de
tempo e memória.

\begin{itemize}
    \item \textbf{Tempo}: $O(1)$.
    \item \textbf{Memória}: $O(1)$.
\end{itemize}

\subsection*{Complexidade Total}

A complexidade total da solução é dominada pela execução do algoritmo de
Edmonds-Karp, que possui complexidade de tempo $O((n \times m)^3)$ e
complexidade de memória $O((n \times m)^2)$. Assim, temos:

\begin{itemize}
    \item \textbf{Tempo Total}: $O((n \times m)^3)$.
    \item \textbf{Memória Total}: $O((n \times m)^2)$.
\end{itemize}

\subsection*{Discussão}

Embora a solução seja eficiente para mapas de tamanho moderado, a complexidade
cúbica em relação ao número de células do mapa pode se tornar um gargalo para
mapas muito grandes. No entanto, a escolha do algoritmo de Edmonds-Karp foi
adequada para o problema, pois garante a correção e permite uma implementação
relativamente simples, atendendo aos requisitos do trabalho prático.

% Capítulo 5: Considerações Finais
\section*{Considerações Finais}

Este trabalho foi uma experiência desafiadora e enriquecedora, permitindo aplicar conceitos de grafos e algoritmos de fluxo máximo.

\subsection*{Partes Mais Fáceis}
A leitura do problema e a implementação da estrutura de dados foram diretas, graças à clareza dos requisitos e à modelagem bem definida.

\subsection*{Partes Mais Difíceis}
A modelagem do grafo e a depuração de erros foram desafiadoras, especialmente para casos de teste complexos. A análise de complexidade também exigiu atenção aos detalhes.

Apesar dos desafios, o trabalho proporcionou um aprendizado significativo e reforçou a importância de uma abordagem sistemática para resolver problemas computacionais.

% Capítulo 6: Referências
\section*{Referências}

As seguintes referências foram utilizadas para o desenvolvimento deste
trabalho:

\begin{itemize}
    \item Slides da disciplina Algoritmos I, disponível no Moodle.
    \item Documentação oficial do Python (\url{https://docs.python.org/3/}).
\end{itemize}

Essas fontes foram fundamentais para compreender os conceitos teóricos e
implementar a solução de forma eficiente e correta.

\end{document}